\section{智能无人机竞赛综合技术培训}

\begin{enumerate}
    \item 结构:架构设计,起落架设计,控制器,传感器安装布局,扩展机构设计
    \item 电子:飞控系统,供电系统,伴随处理器,数据链使用
    \item 导航:位置与姿势,视觉导航,避障与规划
    \item 控制:PID,位置控制,速度控制,姿态控制
    \item 动力:电机,电力,螺旋桨选型,焊接测试
\end{enumerate}

\subsection{智能无人机系统集成}

\begin{enumerate}
    \item 无人机设计与装配
        \subitem  架构设计与加工、无人机设计与装配
        \subitem 飞机动力学
        \subitem 3D打印,焊接
    \item 无人机控制与调试
        \subitem  传感器与组合导航
        \subitem 嵌入式单片机编程
        \subitem 自动化控制原理,地面站
    \item 无人机智能化软件
        \subitem  自主自主定位技术,感知与传感器技术
        \subitem 计算机视觉技术,人工智能技术
        \subitem 机器人操作系统技术,导航与规划技术 
\end{enumerate}

\textcolor{red}{工程能力要求}
\begin{itemize}
    \item 串口测试
    \item 网络测试
    \item Linux操作系统
    \item etc.
\end{itemize}

\begin{figure}[!ht]
	\centering
	\begin{displaymath}
	\mbox{环境}\left\{
	\begin{array}{lr}
	\mbox{基本数据类型} \left\{ \begin{array}{lr}
	\mbox{整型} \left\{ \begin{array}{lr}
	\mbox{整型(int)} \\
	\mbox{短整型(short)} \\
	\mbox{长整型(long)} 
	\end{array} \right.\\
	\mbox{实型} \left\{ \begin{array}{lr}
	\mbox{单精度(float)} \\
	\mbox{双精度(double)} \\
	\mbox{长双精度(long double)} 
	\end{array} \right.\\
	\mbox{字符型(char)}
	\end{array} \right. \\
	\mbox{复杂数据类型} \left\{ \begin{array}{lr}
	\mbox{数组} \\
	\mbox{结构体(int)} \\
	\mbox{联合体(short)} \\
	\mbox{位域} \\
	\mbox{枚举(enum)} 
	\end{array} \right.\\
	\mbox{指针类型} \\
	\mbox{空类型} \\
	\mbox{定义类型} 
	\end{array}
	\right.
	\label{basicdatatype}
	\end{displaymath}
	\caption{C语言数据类型层次结构图}
	\label{basicdatatype02}
\end{figure}


\begin{tikzpicture}[node distance=10pt]
    \node[draw, rounded corners]                        (start)   {Start};
    \node[draw, below=of start]                         (step 1)  {Step 1};
    \node[draw, below=of step 1]                        (step 2)  {Step 2};
    \node[draw, diamond, aspect=2, below=of step 2]     (choice)  {Choice};
    \node[draw, right=30pt of choice]                   (step x)  {Step X};
    \node[draw, rounded corners, below=20pt of choice]  (end)     {End};
    
    \draw[->] (start)  -- (step 1);
    \draw[->] (step 1) -- (step 2);
    \draw[->] (step 2) -- (choice);
    \draw[->] (choice) -- node[left]  {Yes} (end);
    \draw[->] (choice) -- node[above] {No}  (step x);
    \draw[->] (step x) -- (step x|-step 2) -> (step 2);
\end{tikzpicture}